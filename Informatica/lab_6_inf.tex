\documentclass[a4paper,12pt]{article}
\usepackage[utf8]{inputenc}
\usepackage[T2A]{fontenc} % Для корректного отображения кириллицы
\usepackage[russian]{babel}
\usepackage{multicol}
\usepackage{amsmath}
\usepackage{amssymb}
\usepackage{graphicx}
\usepackage{float}
\usepackage[a4paper, left=2cm, right=2cm, top=4cm, bottom=2cm]{geometry}

\author{}
\date{}

\begin{document}

% Заголовок
{\Large \textbf{Квадрат Пирсона}} % Размер заголовка оставлен большим

% Начало текста в двух колонках с увеличенным шрифтом
\begin{multicols}{2}
{\normalsize % Изменено с \small на \normalsize для увеличения шрифта
\textit{А.П. Азия, И.М Вольпер}

\bigskip % Добавлен вертикальный отступ для лучшей читаемости

В «Занимательной алгебре» Я. И. Перельмана есть любопытная задача под названием «В парикмахерской». В задаче автор рассказывает, что, заглянув в парикмахерскую, он увидел, как мастера пытались безуспешно приготовить 12\% раствор перекиси водорода из двух имевшихся в наличии растворов трех и тридцатипроцентного. Задача, описанная Перельманом, встречается не только в парикмахерских. Например, для зарядки аккумуляторов бывает необходимо приготовить электролит, который должен содержать 24\% серной \% слоты из двух растворов с содержанием 92\% и 10\% серной кислоты. На консервных заводах возникает необходимость приготовления 6\% уксуса для маринада из двух партий уксуса разной крепости: 3\% и 10\%, и т. д. Для решения подобных задач удобно пользоваться «квадратом Пирсона». Вот как это делается. Рисуют квадрат и проводят две диагонали (рис. 1). В левом верхнем углу проставляют больший показатель крепости исходных веществ (а), а в нижнем углу второй показатель (6), а на пересечении диагоналей записывают требуемый показатель смеси (с).

Затем производят вычитание по первой диагонали (ас) и находят количество второй части смеси (у). Из центра производят вычитание по второй диагонали (с — б) и находят количество первой части смеси (х).\footnote{*) Напомним, что содержанием вещ-ва в
\\
расстворе называется отношение массы этого
\\
вещ-ва к массе расствора.}

\columnbreak

Значения х и у записывают по одной линии с показателями. На х частей первого вещества надо взять у частей второго вещества, тогда получится смесь с показателем с.

\begin{figure}[H]
    \centering
    \includegraphics[width=0.3\textwidth]{11.png}
    \caption{Пример квадрата Пирсона}
    \label{fig:example}
\end{figure}

Пусть, например, имеются две партии сливок: одна 36\%, а другая 18\%. Требуется определить, сколько надо взять тех и других сливок, чтобы получить смесь с количеством жира 30\%. Решаем по изложенному выше способу (рис.2) и получаем
\begin{align*}
y &= a - c = 36 - 30 = 6 \\
x &= c - b = 30 - 18 = 12
\end{align*}
то есть на 6 массовых частей второй партии сливок надо взять 12 частей первой.

Этот способ основан на специфическом виде количества получаемой смеси, оно равно разности показателей исходных веществ. Такое допущение вполне возможно, так как нас интересуют не абсолютные величины, а относительные количества двух частей смеси.

В самом деле, мы получаем
$$
x + y = (c - b) + (a - c) = a - b
$$
частей смеси. "Чистого" вещества в ней будет
$$
\frac{x \cdot a}{100} + \frac{y \cdot b}{100} = \frac{(c - b) \cdot a + (a - c) \cdot b}{100} = \frac{a \cdot c - a \cdot b}{100}
$$
частей, а крепость смеси будет равна
$$
\frac{ac - bc}{100(a-b)} = \frac{c}{100}
$$, то есть с \%
} % Конец увеличенного шрифта
\end{multicols}

\end{document}